\documentclass[pdf]{beamer}

%% Config
\usepackage{fontspec}
\usepackage{polyglossia}
\setmainlanguage{ukrainian}
\setotherlanguage{english}
\newfontfamily\cyrillicfont[Script=Cyrillic]{CMU Serif}
\newfontfamily\cyrillicfontsf[Script=Cyrillic]{CMU Sans Serif}
\newfontfamily\cyrillicfonttt[Script=Cyrillic]{CMU Typewriter Text}
 
% Useful packages
\usepackage{amssymb, amsmath, amsfonts, enumerate, float, indentfirst, graphicx}
\usepackage[final]{listings}

\usepackage{tikz}

% Listings
\lstdefinestyle{mystyle}{
  language=python,
  basicstyle=\ttfamily\small,
  showstringspaces=false,
  belowskip=1em,
  aboveskip=1em
%  frame=single,
}
\lstset{escapechar=@,style=mystyle}

%% Title
\title{Застосування марківських випадкових полів для моделювання економічних процесів з конкурентними технологіями}
\date[2014]{14 травня 2014}
\author{Богдан Кулинич}

%% Theme
\usetheme{boxes}
\usecolortheme{seahorse}

\setbeamercovered{transparent=20}

\begin{document}

\begin{frame}
\titlepage
\end{frame}


\begin{frame}{Задача}

\begin{itemize}
\item Система економічних агентів (підприємств), що взаємодіють між собою.
\item У кожен момент часу підприємство обирає технологію виробництва
\item Вибір технології впливає на стан системи
\item Кожна технологія несе певні витрати (або приносить доходи) для підприємства
\end{itemize}

\begin{block}{ }
Знайти стратегію вибору технологій, що мінімізує очікувані витрати (або максимізує очікувані доходи)
\end{block}
	
\end{frame}


\begin{frame}{Модель системи}{Формулювання}
	
	\begin{center}
	\begin{tikzpicture}
	  [scale=.2,auto=center,every node/.style={circle,fill=blue!20}]
	  \node (a) at (0,10) {A};
	  \node (b) at (8,10) {B};
	  \node (d) at (16,10) {D};
	  \node (c) at (4, 4) {C};

	  \foreach \from/\to in {a/b, b/c, a/c, b/d}
		\draw (\from) -- (\to);

	\end{tikzpicture}
	\end{center}
	
	\begin{itemize}
	\item Множина економічних агентів \(V = \{A, B, C, D\}\)
	\item Граф взаємодії агентів \(G = (V, E)\)
	\end{itemize}
	
\end{frame}


\begin{frame}{Модель системи}{Формулювання}
	\begin{itemize}[<+->]
	\item Стани \(X_v\)
	\begin{gather*}
		X_A = \{a_1, a_2, a_3\}, X_B = \{b_1, b_2\} \\ 
		X_C = \{c_1, c_2\}, X_D = \{c_1, c_2, c_3\}
	\end{gather*}
	\(X = \times_{v \in V} X_v\) – системи. \(|X| = 3 \cdot 2 \cdot 2 \cdot 3 = 36 \)
	\vspace{1em}
	\item Можливі дії агентів (\textit{технології виробництва}) \(U_v\)
	\begin{gather*}
		U_A = \{\alpha_1, \alpha_2\}, U_B = \{\beta_1, \beta_2, \beta_3\} \\ 
		U_C = \{\gamma_1, \gamma_2, \gamma_3, \gamma_4\}, U_D = \{\zeta_1, \zeta_2\}
	\end{gather*}
	\(U = \times_{v \in V} U_v\) – дії системи. \(|U| = 2 \cdot 3 \cdot 4 \cdot 2 = 48\)
	\end{itemize}
\end{frame}


\begin{frame}{Модель системи}{Формулювання}
	\begin{itemize}
	\item<1-> Еволюція системи відбувається у дискретному часі \(t = 0, 1, \ldots\)
	\item<2-> Зміна станів відбувається стохастично. \\
	\((\Omega, \mathcal{F}, P).\quad \xi = (\xi^t\ |\ t = 0, 1, \ldots).\quad \xi^t : \Omega \rightarrow X\)
	\item<3-> Рішення про вибір технологій \(\Delta^t: X \rightarrow U \) \\
	\item<3-> Cтаціонарні стратегії \(\delta = \{\Delta^t\ |\ t=0, 1, \ldots\}\)
	\item<4-> Кожне рішення має витрати (або дохід) \(r: X \times U \rightarrow \mathbb{R} \)
	\end{itemize}
\end{frame}


\begin{frame}{Модель системи}{Основні припущення}
\begin{block}{Означення}
Нехай закритий окіл вершини графа \(G = (V, E)\): \(\tilde{N}(v) = \big\{w\ |\ \{v, w\} \in E\big\} \cup \{v\}\:\)
\end{block}

	\begin{itemize}
	\item<2-> \textit{Локальність} взаємодії агентів відносно графа \(G\)
    \begin{align*}
    P(\xi_v^{t+1} = x_v\ &|\ \xi^0 = x^0, \Delta^0 = u^0,\ \ldots, \xi^t = x^t, \Delta^t = u^t) = \\
	= P(\xi_v^{t+1} = x_v\ &|\ \xi_{\tilde N(v)}^0 = x_{\tilde N(v)}^0, \Delta_v^0 = u_v^0,\ \ldots,\\ &\xi_{\tilde N(v)}^t = x_{\tilde N(v)}^t, \Delta_v^t = u_v^t)
	\end{align*}
	\item<3-> \textit{Локальність} рішень відносно графа \(G\)
	\[\Delta_v^{t} = \Delta_v^{t}(x^0, x^1, \ldots, x^t) = \Delta_v^{t}(x_{\tilde{N}(v)}^0, x_{\tilde{N}(v)}^1, \ldots, x_{\tilde{N}(v)}^t)\]
	\end{itemize}

\end{frame}


\begin{frame}{Модель системи}{Основні припущення}
\begin{itemize}[<+->]
	\item \textit{Марковість} (повнота стану)
    \begin{align*}
    P(\xi_v^{t+1} = x_v\ &|\ \xi^0 = x^0, \Delta^0 = u^0,\ \ldots, \xi^t = x^t, \Delta^t = u^t) = \\
	= P(\xi_v^{t+1} = x_v\ &|\ \xi_{\tilde N(v)}^t = x_{\tilde N(v)}^t, \Delta_v^t = u_v^t) \\
	&\Delta_v^{t} = \Delta_v^{t}(x_{\tilde{N}(v)}^t)
	\end{align*}
	\item \textit{Синхронність} взаємодії
	\begin{align*}
	&P(\xi^{t+1}_W = x_W\ |\ \xi^t = x^t, \Delta^t = u^t) = \\
    = &\prod_{w \in W} P(\xi^{t+1}_w = x_w\ |\ \xi^t = x^t, \Delta^t = u^t)
    \end{align*}
	\end{itemize}
\end{frame}


\begin{frame}{Модель системи}{Приклад}
	\begin{center}
	\begin{tikzpicture}
	  [scale=.2,auto=center,every node/.style={circle,fill=blue!20}]
	  \node (a) at (0,10) {A};
	  \node (b) at (8,10) {B};
	  \node (d) at (16,10) [fill=red!60] {D};
	  \node (c) at (4, 4) {C};

	  \foreach \from/\to in {a/b, b/c, a/c}
		\draw (\from) -- (\to);
	  \draw [red!50] (b) -- (d);

	\end{tikzpicture}
	\end{center}
	
	\begin{gather*}
	P(\xi_D^{t+1} = x\ |\xi^0 = x^0, \Delta^0 = u^0, \ldots, \xi^t = x^t, \Delta^t = u^t) \\
	= P(\xi_D^{t+1} = x\ |\ \xi_D = d, \xi_B = b, \Delta_D^t = u) \\
	\\
	\Delta_D^{t} = \Delta_D^{t}(\xi_D^t,\ \xi_B^t)
	\end{gather*}
\end{frame}


\begin{frame}{Модель системи}{Задання}

Однозначно задають систему:
\begin{itemize}
	\item \(R\) – матриця витрат, \(R_{ij} = r(x_i, u_j)\). Розмірність \(|X| \times |U| = 36 \times 48\)
	\item \(Q^k\) – матриці перехідних імовірностей. 
	\[Q_{ij}^k = P(\xi^{t+1} = x_j\ |\ \xi^t = x_i, \Delta^t = u_k)\]
	Розмірність кожної \(Q^k\) – \(|X| \times |X| = 36 \times 36\). Кількість \(Q^k\) – \(|U| = 48\)
\end{itemize}

\end{frame}


\begin{frame}{Модель системи}{Марківське випадкове поле}
\begin{block}{Означення}
Визначений керований процес \((\xi, \delta)\) відносно графа взаємодії \(G = (V, E)\) називається \textit{керованим марківським випадковим полем з дискретним часом}.
\end{block}
\begin{itemize}
\item<2->{За стаціонарності стратегій, таке поле також є однорідним марківським ланцюгом}
\end{itemize}
\end{frame}


\begin{frame}{Знаходження оптимальної стратегії}{Формулювання}
	\begin{itemize}[<+->]
	\item Математичне сподівання витрат для стратегії \(\delta\):
	\[ C_T^{\delta} = E\left[\frac{1}{T+1}\sum_{t=0}^T r(\xi^t, \Delta^t)\right] \]
	\item Оптимальна стратегія \(\delta^*\) мінімізує \(C_T^{\delta}(y)\) при \(T\rightarrow\infty\) для всіх \(y \in X\):
	\begin{gather*}
	R_y^\delta = \lim\limits_{T \rightarrow \infty} \sup C_T^{\delta}(y) \\
	\delta^* = R_y^{\delta^*},\ y \in X
	\end{gather*}
	\end{itemize}
\end{frame}


\begin{frame}{Знаходження оптимальної стратегії}{Задача лінійного програмування}

Для даного керованого марківського поля стратегію \(\delta^*\) можна знайти, розв’язавши задачу ЛП:

\[\min \sum_{x \in X} \sum_{u \in U} r(x, u)\ z_{xu}\]
З обмеженнями:
\begin{gather*}
\sum_{u \in U} z_{xu} = \sum_{y \in X} \sum_{u \in U} Q_{yx}^u\,z_{yu},\ x \in X \\
\sum_{x \in X} \sum_{u \in U} z_{xu} = 1 \\
z_{xu} \geq 0,\ x \in X, u \in U
\end{gather*}

\end{frame}


\begin{frame}{Знаходження оптимальної стратегії}{Інтерпретація}
Стратегія \(\delta^*\) \textit{нерандомізована} і задається таким чином:
\begin{itemize}
	\item Якщо \(z_{xu} \neq 0\), то в стані \(x\) слід обирати дію \(u\)
	\item Якщо \(z_{xu} = 0\), то в стані \(x\) \textit{не слід} обирати дію \(u\)
\end{itemize}
\end{frame}

\begin{frame}{Fin}
Програмна реалізація, код та сама курсова робота:

\begin{center}
	\url{github.com/bogdan-kulynych/mrf-in-economics}
\end{center}

\end{frame}

\end{document}
