\documentclass[oneside,draft,14pt]{extarticle}

% Config
% Ukrainian language

\usepackage{fontspec}
\usepackage{polyglossia}
\setmainlanguage{ukrainian}
\setotherlanguage{english}
\newfontfamily\cyrillicfont[Script=Cyrillic]{CMU Serif}
\newfontfamily\cyrillicfontsf[Script=Cyrillic]{CMU Sans Serif}
\newfontfamily\cyrillicfonttt[Script=Cyrillic]{CMU Typewriter Text}

% Page setup
\usepackage{vmargin}
\setpapersize{A4}
\setmarginsrb{2cm}{1.5cm}{1cm}{1.5cm}{0pt}{0mm}{0pt}{13mm}
\renewcommand{\baselinestretch}{1.4}
\sloppy

% Useful packages
\usepackage{amssymb, amsmath, amsfonts, enumerate, float, indentfirst}

% Bibliography
\usepackage{cite}
%\usepackage[nottoc, numbib]{tocbibind}
\makeatletter
\renewcommand{\@biblabel}[1]{#1.}
\makeatother

% Arabic enumerations
\renewcommand{\theenumi}{\arabic{enumi}}
\renewcommand{\labelenumi}{\arabic{enumi}}
\renewcommand{\theenumii}{.\arabic{enumii}}
\renewcommand{\labelenumii}{\arabic{enumi}.\arabic{enumii}.}
\renewcommand{\theenumiii}{.\arabic{enumiii}}
\renewcommand{\labelenumiii}{\arabic{enumi}.\arabic{enumii}.\arabic{enumiii}.}

% Hide sections numbers
% http://tex.stackexchange.com/questions/80113/hide-section-numbers-but-keep-numbering
\renewcommand{\thesection}{}
\renewcommand{\thesubsection}{\arabic{section}.\arabic{subsection}}
\makeatletter
\def\@seccntformat#1{\csname #1ignore\expandafter\endcsname\csname the#1\endcsname\quad}
\let\sectionignore\@gobbletwo
\let\latex@numberline\numberline
\def\numberline#1{\if\relax#1\relax\else\latex@numberline{#1}\fi}
\makeatother

% Theorems
\usepackage{amsthm}
\theoremstyle{definition}
\newtheorem*{definition}{Означення}
\newtheorem*{example}{Приклад}
\newtheorem*{algorithm}{Алгоритм}

\begin{document}

% Title page
\begin{titlepage}
\newpage

\begin{center}
Міністерство освіти і науки України \\
\textsc{Національний Університет «Києво-Могилянська Академія»} \\
Кафедра математики факультету інформатики
\end{center}

\vspace{8em}

\vspace{2em}

\begin{center}
\textsc{\textbf{Застосування Марківських випадкових полів для моделювання економічних процесів з конкурентними технологіями}} \\
Курсова робота за спеціальністю «Прикладна математика»
\end{center}

\vspace{6em}

\begin{flushright}
Керівник курсової роботи \\
к. фіз.-мат. наук, доцент \\
Чорней Р.К. \\
\vspace{2em}
Виконав студент \\
Кулинич Б.В. \\
\vspace{2em}

\end{flushright}

\vspace{\fill}

\begin{center}
Київ – 2014
\end{center}

\end{titlepage}
\setcounter{page}{2}

% Toc
\tableofcontents
\clearpage

% Contents
\section{Вступ}

Два абзаци про все це діло.

\section{Процеси з конкурентними категоріями}

\subsection{Постановка задачі}

Описана у цій роботі задача виникає в економіці\cite{David:1998} при моделюванні поведінки систем підприємств, що взаємодіють між собою, причому основним фактором взаємодії є обрані кожним із підприємств технології виробництва. Модель має відповідати на питання, яку технологію слід обирати кожному з підприємств у кожний момент часу, щоб мінімізувати свої витрати і, відповідно, максимізувати дохід. 

У загальному випадку, система описується\cite{Chornei:2005} множиною економічних агентів \(V\), неорієнтованим графом їхніх взаємодій \(G = (V, E)\), множиною \(X^t = \times_{v \in V}X_v^t,\ X_v^t = \{x^{t,1}_v, x^{t,2}_v, \ldots, x^{t,n_v}_v\}\) можливих станів (технологій виробництва) агента \(v \in V\) у момент часу \(t\), та послідновностями рішень (про вибір відповідних технологій) \(\Delta_v^{t+1} = \Delta_v^{t+1}(x^0, x^1, \ldots, x^t) \in U_v^t = X_v^t\) агента \(v\) у момент часу \(t\).

Позначимо як \(N(i)\) множину агентів, що взаємодіють з агентом \(i \in V\), і як \(\tilde{N}(i)\) — множину \(N(i)\) разом із самим \(i\):
\(N(i) := \{j\ |\ \{i, j\} \in E\},\ \tilde{N}(i) := N(i) \cup \{i\}\).

Надалі, для побудови ефективної моделі, щодо системи буде введено ряд обмежень.

\begin{description} 
    \item[Дискретний час.] Еволюція системи відбувається у дискретному часі, тобто \(t = 0, 1, ...\,\). В контексті заданої задачі, припущення має сенс, оскільки зручно розглядати дії підприємств в регулярні дискретні моменти часу, наприклад, щодня, щомісяця, щоквартала тощо.
    \item[Синхронність.] Усі агенти системи переходять у свій наступний стан одночасно. Це припущення узгоджується з інтерпретацією \(t\) як конкретного моменту в часі, однакового для всіх агентів.
    \item[Інваріантність простору станів.] Для всіх \(v \in V\) у кожен момент часу \(t\) \(X_v^t = X_v\). Природньо, що в реальних системах набір доступних технологій змінюється рідко (внаслідок інновацій, або застаріння технологій), тому змінністю простору станів можна знехтувати в багатьох випадках. Якщо ж знехтувати не можна, достатньо розглянути випадки як окремі системи з різними просторами станів.
    \item[Стохастичність.] Для всіх \(v \in V\)
    \[ \xi_v = \{\xi^t_v\ |\ t = 0, 1, \ldots \}\] \[ \xi_v^t : (\Omega, \mathcal{F}, P) \rightarrow X_v,\]
    де \( \xi_v \) – стохастичний процес, що визначає стан \(v\) у кожен момент часу \(t\). Припущення дає змогу працювати з системою засобами теорії ймовірностей.
    \item[Локальність.] Для всіх \(v \in V\) 
    \[\Delta_v^{t+1} = \Delta_v^{t+1}(x^0, x^1, \ldots, x^t) = \Delta_v^{t+1}(x_{\tilde{N}(v)}^0, x_{\tilde{N}(v)}^1, \ldots, x_{\tilde{N}(v)}^t)\]
    \begin{align*}
    P(\xi_v^{t+1} = x_v\ &|\ \xi_{V\setminus\{v\}}^0 = x^0, \Delta_{V\setminus\{v\}}^0 = u^0,\ \ldots, \xi_{V\setminus\{v\}}^t = x^t, \Delta_{V\setminus\{v\}}^t = u^t) = \\
	= P(\xi_v^{t+1} = x_v\ &|\ \xi_{\tilde N(v)}^0 = x_{\tilde N(v)}^0, \Delta_{\tilde N(v)}^0 = u_{\tilde N(v)}^0,\ \ldots,\ \xi_{\tilde N(v)}^t = x_{\tilde N(v)}^t, \Delta_{\tilde N(v)}^t = u_{\tilde N(v)}^t)
	%= P(\xi_v^{t+1} = x_v\ |\ \xi_{\tilde N(v)}^t = x_{\tilde N(v)}^t, \Delta_{\tilde N(v)}^t = u_{\tilde N(v)}^t)
	\end{align*}
	Локальність є природньою в контексті задачі, оскільки за визначенням системи, агенти, що не взаємодіють з даним, не чинять на нього безпосереднього впливу. Звідси, агентові для прийняття рішення достатньо знати стани тільки тих агентів, з якими він взаємодіє.
	\item[Повнота стану.] (Умова Маркова) Для всіх \(v \in V\)
	\[\Delta_v^{t+1}(x_{\tilde{N}(v)}^0, x_{\tilde{N}(v)}^1, \ldots, x_{\tilde{N}(v)}^t) = \Delta_v^{t+1}(x_{\tilde{N}(v)}^t)\]
	\begin{align*}
	P(\xi_v^{t+1} = x_v\ &|\ \xi_{\tilde N(v)}^0 = x_{\tilde N(v)}^0, \Delta_{\tilde N(v)}^0 = u_{\tilde N(v)}^0,\ \ldots,\ \xi_{\tilde N(v)}^t = x_{\tilde N(v)}^t, \Delta_{\tilde N(v)}^t = u_{\tilde N(v)}^t) = \\
	= P(\xi_v^{t+1} = x_v\ &|\ \xi_{\tilde N(v)}^t = x_{\tilde N(v)}^t, \Delta_{\tilde N(v)}^t = u_{\tilde N(v)}^t)
	\end{align*}
	Припущення є реалістичним, оскільки, насправді, найважливішим є стан сусідніх агентів у попередній момент часу, тоді як повною історією їх рішень можна знехтувати.
\end{description}

Позначимо \(\delta := \{\delta_v\ |\ v \in V\}\), де \(\delta_v = \{\Delta_v^t, t \geq 0\}\)

З усіма вказаними припущеннями, визначена модель є Марківським випадковим полем з дискретним часом відносно графа взаємозв’язків \(G = (V, E)\), і керованого стохастичного процесу \((\xi, \delta)\) з локальною стратегією \(\delta\). Розглянемо цей клас об’єктів детальніше.

\subsection{Марківські випадкові поля}
\begin{definition}
Неорієнтований граф \(G = (V, E)\), множина випадкових величин \(\xi_v : (\Omega, \mathcal{F}, P) \rightarrow X,\ v \in V\) утворюють \textit{Марківське випадкове поле}, якщо для всіх \(v \in V\):
\[P(\xi_v = x_v\ |\ \xi_{V\setminus\{v\}} = x_{V\setminus\{v\}}) = P(\xi_k = x_v\ |\ \xi_{N(v)} = x_{N(v)})\]
\end{definition}

\begin{definition}
Нехай \(\xi = \{\xi^t\ |\ t=0,1,\ldots\},\ \xi^t_v : (\Omega, \mathcal{F}, P) \rightarrow X_v,\ v \in V \) — марківський процес з дискретним часом. 
Якщо виконуються:
\begin{description}
    \item[Локальність] \[P(\xi_v^{t+1} = x_v\ |\ \xi_v^0 = x^0, \xi_v^1 = x^1, \ldots, \xi_v^t = x^t) = P(\xi_v^{t+1} = x_v\ |\ \xi_{\tilde N(v)}^t = x_{\tilde N(v)}^t )\]
    \item[Синхронність] \[P(\xi^{t+1}_K = x_K\ |\ \xi^t = x^t) = \prod_{k \in K} P(\xi^{t+1}_k = x_k\ |\ \xi^t = x^t), K \subset V\]

Тоді процес \(\xi\) разом із графом \(G = (V, E)\) утворює \textit{Марківське випадкове поле із синхронними компонентами, що локально взаємодіють}, або просто \textit{Марківське випадкове поле з дискретним часом}.
\end{description}
\end{definition}


\clearpage

% Literature
\nocite{David:1998}
\nocite{Knopov:2011}
\nocite{Knopov:1998}
\nocite{Chornei:2005}
\nocite{Koller:2009}

\bibliography{literature.bib}
\bibliographystyle{abbrv}

\end{document}
