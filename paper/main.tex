\documentclass[oneside,draft,14pt]{extarticle}

% Config
% Ukrainian language

\usepackage{fontspec}
\usepackage{polyglossia}
\setmainlanguage{ukrainian}
\setotherlanguage{english}
\newfontfamily\cyrillicfont[Script=Cyrillic]{CMU Serif}
\newfontfamily\cyrillicfontsf[Script=Cyrillic]{CMU Sans Serif}
\newfontfamily\cyrillicfonttt[Script=Cyrillic]{CMU Typewriter Text}

% Page setup
\usepackage{vmargin}
\setpapersize{A4}
\setmarginsrb{2cm}{1.5cm}{1cm}{1.5cm}{0pt}{0mm}{0pt}{13mm}
\renewcommand{\baselinestretch}{1.4}
\sloppy

% Useful packages
\usepackage{amssymb, amsmath, amsfonts, enumerate, float, indentfirst}

% Bibliography
\usepackage{cite}
%\usepackage[nottoc, numbib]{tocbibind}
\makeatletter
\renewcommand{\@biblabel}[1]{#1.}
\makeatother

% Arabic enumerations
\renewcommand{\theenumi}{\arabic{enumi}}
\renewcommand{\labelenumi}{\arabic{enumi}}
\renewcommand{\theenumii}{.\arabic{enumii}}
\renewcommand{\labelenumii}{\arabic{enumi}.\arabic{enumii}.}
\renewcommand{\theenumiii}{.\arabic{enumiii}}
\renewcommand{\labelenumiii}{\arabic{enumi}.\arabic{enumii}.\arabic{enumiii}.}

% Hide sections numbers
% http://tex.stackexchange.com/questions/80113/hide-section-numbers-but-keep-numbering
\renewcommand{\thesection}{}
\renewcommand{\thesubsection}{\arabic{section}.\arabic{subsection}}
\makeatletter
\def\@seccntformat#1{\csname #1ignore\expandafter\endcsname\csname the#1\endcsname\quad}
\let\sectionignore\@gobbletwo
\let\latex@numberline\numberline
\def\numberline#1{\if\relax#1\relax\else\latex@numberline{#1}\fi}
\makeatother

% Theorems
\usepackage{amsthm}
\theoremstyle{definition}
\newtheorem*{definition}{Означення}
\newtheorem*{example}{Приклад}
\newtheorem*{algorithm}{Алгоритм}

\begin{document}

% Title page
\begin{titlepage}
\newpage

\begin{center}
Міністерство освіти і науки України \\
\textsc{Національний Університет «Києво-Могилянська Академія»} \\
Кафедра математики факультету інформатики
\end{center}

\vspace{8em}

\vspace{2em}

\begin{center}
\textsc{\textbf{Застосування Марківських випадкових полів для моделювання економічних процесів з конкурентними технологіями}} \\
Курсова робота за спеціальністю «Прикладна математика»
\end{center}

\vspace{6em}

\begin{flushright}
Керівник курсової роботи \\
к. фіз.-мат. наук, доцент \\
Чорней Р.К. \\
\vspace{2em}
Виконав студент \\
Кулинич Б.В. \\
\vspace{2em}

\end{flushright}

\vspace{\fill}

\begin{center}
Київ – 2014
\end{center}

\end{titlepage}
\setcounter{page}{2}

% Toc
\tableofcontents
\clearpage

% Contents
\section{Вступ}

В даній роботі описана задача, що виникає в економіці\cite{David:1998} при моделюванні поведінки систем підприємств, що взаємодіють між собою. В кожен момент часу підприємства можуть обирати технологію виробництва, яка впливає на стан підприємства, а також на поведінку інших підприємств в наступні моменти часу.

Метою роботи є побудова ефективної моделі такої системи, що дозволить отримувати відповідь на питання, яку технологію слід обирати кожному з підприємств у кожний момент часу, щоб мінімізувати свої витрати (або максимізувати дохід).

\section{Економічні процеси з конкурентними технологіями}

\subsection{Постановка задачі}
\label{sec:description}

У загальному випадку система описується\cite{Chornei:2005} для дискретного часу \(t = 0, 1, \ldots \) скінченною множиною економічних агентів \(V\) і неорієнтованим графом їхніх взаємодій \(G = (V, E)\), скінченною множиною \(X^t = \times_{v \in V}X_v^t,\ X_v^t = \{x^{t,1}_v, x^{t,2}_v, \ldots, x^{t,n_v}_v\}\) можливих станів агента \(v \in V\) у момент часу \(t\), множиною рішень (про вибір відповідних технологій) \(\Delta_v^t: \times_{t=0,1,\ldots}X^t \rightarrow U_v^t\), де \(U_v^t\) – скінченна множина можливих дій, та відповідних їм функцій витрат (доходів) \(r_v^t: X_v^t \times U_v^t \rightarrow \{u\ |\ u \in \mathbb{R}, |u| < C\}\) для агента \(v \in V\) у момент часу~\(t\).

%\subsubsection{Припущення}

Для побудови ефективної моделі, щодо системи буде додатково введено ряд обмежень.

\begin{description}
    \item[Інваріантність рішень, та просторів станів і можливих дій] Для всіх \(v \in V\), у кожен момент часу \(t', t'' = 0, 1, ...\ \)
    \begin{align*}
      X_v^{t'} &= X_v^{t''} = X_v \\
      U_v^{t'} &= U_v^{t''} = U_v \\
	  \Delta_v^{t'} &= \Delta_v^{t''} = \Delta_v
    \end{align*}
	Природньо, що в реальних системах набір доступних технологій змінюється рідко (внаслідок інновацій, або застаріння технологій), тому змінністю простору можливих дій можна знехтувати в багатьох випадках, так само, як і змінністю простору станів.
	
	Вважитемо також, що в кожен момент часу \(t\) для кожного агента \(v \in V\) всі можливі дії \(U_v^t\) допустимі.

    \item[Стохастичність.] Для всіх \(v \in V\)
    \[ \xi_v = \{\xi^t_v\ |\ t = 0, 1, \ldots \}\] 
    \[ (\Omega, \mathcal{F}, P).\quad \xi_v^t : X_v \rightarrow [0, 1]\ ,\]
    де \( \xi_v \) – стохастичний процес, що визначає стан \(v\) у кожен момент часу \(t\). Припущення дає змогу працювати з системою засобами теорії ймовірностей.
    
    \item[Локальність.] Для всіх \(v \in V\) 
    \[\Delta_v^{t+1} = \Delta_v^{t+1}(x^0, x^1, \ldots, x^t) = \Delta_v^{t+1}(x_{\tilde{N}(v)}^0, x_{\tilde{N}(v)}^1, \ldots, x_{\tilde{N}(v)}^t)\]
    \begin{align*}
    P(\xi_v^{t+1} = x_v\ &|\ \xi^0 = x^0, \Delta^0 = u^0,\ \ldots, \xi^t = x^t, \Delta^t = u^t) = \\
	= P(\xi_v^{t+1} = x_v\ &|\ \xi_{\tilde N(v)}^0 = x_{\tilde N(v)}^0, \Delta_{\tilde N(v)}^0 = u_{\tilde N(v)}^0,\ \ldots,\\ &\xi_{\tilde N(v)}^t = x_{\tilde N(v)}^t, \Delta_{\tilde N(v)}^t = u_{\tilde N(v)}^t)
	\end{align*}
	Локальність є природньою в контексті задачі, оскільки за визначенням системи, агенти, що не взаємодіють з даним, не чинять на нього безпосереднього впливу. Звідси, агентові для прийняття рішення достатньо знати стани тільки тих агентів, з якими він взаємодіє.
	
    \item[Синхронність.] Для всіх \(W \subset V\)
     \[P(\xi^{t+1}_W = x_W\ |\ \xi^t = x^t, \Delta^t = u^t) = \prod_{w \in W} P(\xi^{t+1}_w = x_w\ |\ \xi^t = x^t, \Delta^t = u^t) \] Усі агенти системи переходять у свій наступний стан одночасно. Це припущення узгоджується з інтерпретацією \(t\) як конкретного моменту в часі, однакового для всіх агентів.
      
	\item[Повнота стану.] (припущення Маркова) Для всіх \(v \in V\)
	\[\Delta_v^{t+1}(x_{\tilde{N}(v)}^0, x_{\tilde{N}(v)}^1, \ldots, x_{\tilde{N}(v)}^t) = \Delta_v^{t+1}(x_{\tilde{N}(v)}^t)\]
	\begin{align*}
	P(\xi_v^{t+1} = x_v\ &|\ \xi_{\tilde N(v)}^0 = x_{\tilde N(v)}^0, \Delta_{\tilde N(v)}^0 = u_{\tilde N(v)}^0,\ \ldots,\\ &\xi_{\tilde N(v)}^t = x_{\tilde N(v)}^t, \Delta_{\tilde N(v)}^t = u_{\tilde N(v)}^t) = \\
	= P(\xi_v^{t+1} = x_v\ & |\ \xi_{\tilde N(v)}^t = x_{\tilde N(v)}^t, \Delta_{\tilde N(v)}^t = u_{\tilde N(v)}^t)
	\end{align*}
	Припущення є реалістичним, оскільки для прийняття рішення про вибір технології найважливішим є стан сусідніх агентів у попередній момент часу, тоді як повною історією їх рішень можна знехтувати.

\end{description}






%\subsubsection{Модель системи}

Визначимо ряд понять, які будуть використані при побудові моделі системи.

Для неорієнтованого графа \(G = (V, E)\) позначимо як \(N(v)\) множину вершин, що з’єднані з вершиною \(v \in V\), і як \(\tilde{N}(v)\) — множину \(N(v)\) разом із самою \(v\):
\[N(v) = \{w\ |\ \{v, w\} \in E\},\ \tilde{N}(v) = N(v) \cup \{v\}\]

\begin{definition}
Неорієнтований граф \(G = (V, E)\), множина випадкових величин, визначених на імовірнісному просторі \((\Omega, \mathcal{F}, P),\ \xi_v : X_v \rightarrow [0, 1]\ ,\ v \in V\) утворюють \textit{марківське випадкове поле}, якщо для всіх \(v \in V\):
\[P(\xi_v = x_v\ |\ \xi_{V\setminus\{v\}} = x_{V\setminus\{v\}}) = P(\xi_k = x_v\ |\ \xi_{N(v)} = x_{N(v)})\]
\end{definition}

Поняття можна розширити до стохастичних процесів.
\begin{definition}
\label{def:mrftime}
Нехай \(\xi = \{\xi^t\ |\ t=0,1,\ldots\},\ v \in V \) — марківський процес з дискретним часом. 
Якщо виконуються:

\begin{enumerate}
    \item (Локальність) Для всіх \(v \in V\) \[P(\xi_v^{t+1} = x_v\ |\ \xi^0 = x^0, \xi^1 = x^1, \ldots, \xi^t = x^t) = P(\xi_v^{t+1} = x_v\ |\ \xi_{\tilde N(v)}^t = x_{\tilde N(v)}^t )\]
    \item (Синхронність) Для всіх \(W \subset V\) \[P(\xi^{t+1}_W = x_W\ |\ \xi^t = x^t) = \prod_{w \in W} P(\xi^{t+1}_w = x_w\ |\ \xi^t = x^t) \]
\end{enumerate}

Тоді процес \(\xi\) разом із графом \(G = (V, E)\) утворює \textit{марківське випадкове поле із синхронними компонентами, що локально взаємодіють}, або \textit{марківське випадкове поле з дискретним часом}. 
\end{definition}

Відразу з означення маємо, що для будь-якого \(W \subset V\)
\[P(\xi_W^{t+1} = x_W^{t+1}\ |\ \xi^t = x^t) = \prod_{w \in W} P(\xi_w^{t+1} = x_w^{t+1}\ |\ \xi_{\tilde{N}(w)}^t = x_{\tilde{N}(w)}^t)\]

Нехай \(V\) – скінченна множина агентів, що приймають рішення. \(U_v\) — скінченний простір можливих дій для агента \(v \in V\), причому \(U_v\) незалежний від часу, \(\Delta_v^t: \times_{t=0,1,\ldots}X^t \rightarrow U_v\) – рішення агента \(v\), залежні від історії станів. 

\begin{definition}
Стратегія \(\delta = \{\delta_v\ |\ v \in V\}\), де \(\delta_v = \{\Delta_v^t\ |\ t=0,1,\ldots\}\), називається локальною, якщо для всіх \(v \in V,\ t = 0, 1, \ldots\)
\[\Delta_v^t = \Delta_v^t(x_{\tilde{N}(v)}^0, \ldots, x_{\tilde{N}(v)}^t)\]
\end{definition}

\begin{definition}
Локальна стратегія називається марківською, якщо для всіх \(v \in V,\ t = 0, 1, \ldots\) 
\[\Delta_v^{t+1} = \Delta_v^{t+1}(x_{\tilde{N}(v)}^t)\]
\end{definition}

\begin{definition}
Cтратегія називається стаціонарною, якщо для всіх \(v~\in~V,\ t', t'' = 0, 1, \ldots,\)
\[\Delta_v^{t'} = \Delta_v^{t''}\]
\end{definition}

Якщо застосувати локальну марківську стратегію \(\delta\) до марківського випадкового поля \(\xi\) відносно графа \(G\), то пара \((\xi, \delta)\) утворює керований марківський ланцюг. Якщо при цьому стратегія \(\delta\) – стаціонарна, то ланцюг однорідний.

	Аналогічно до означення \ref{def:mrftime}, визначимо контрольоване марківське випадкове поле у часі.
	
\begin{definition}
\label{def:ctrlmrftime}
Нехай (\(\xi, \delta \)) — керований стохастичний процес із графом взаємодії \(G = (V, E)\), де \(\delta\) – локальна марківська стратегія. Якщо виконується:
\begin{enumerate}
    \item (Локальність) Для всіх \(v \in V\)
	\begin{align*}
    P(\xi_v^{t+1} = x_v\ &|\ \xi^0 = x^0, \Delta^0 = u^0,\ \ldots, \xi^t = x^t, \Delta^t = u^t) = \\
	= P(\xi_v^{t+1} = x_v\ &|\ \xi_{\tilde N(v)}^t = x_{\tilde N(v)}^t, \Delta_{\tilde N(v)}^t = u_{\tilde N(v)}^t)
	\end{align*}
	\item (Синхронність) Для всіх \(W \subset V\)
	\[P(\xi^{t+1}_W = x_W\ |\ \xi^t = x^t, \Delta^t = u) = \prod_{w \in W} P(\xi^{t+1}_w = x_w\ |\ \xi^t = x^t, \Delta^t = u) \]
\end{enumerate}
Тоді \((\xi, \delta)\) разом із графом \(G=(V,E)\) утворює \textit{кероване марківське поле з дискретним часом}.
\end{definition}

Як наслідок з умов означення, маємо для будь-якого \(W \subset V\)
\begin{align*}
&P(\xi_W^{t+1} = x_W\ |\ \xi^t = x^t, \Delta^t = u^t) = \\
= &\prod_{w \in W} P(\xi^{t+1} = x_w\ |\ \xi_{\tilde{N}(w)}^t = y_{\tilde{N}(w)}, \Delta_w^t(\xi_{\tilde{N}(w)}^t) = u_w)
\end{align*}

\begin{definition} 
Позначимо
\begin{align*}
Q_w(x_w / y_{\tilde{N}(w)}, u_w) &= P(\xi^{t+1} = x_w\ |\ \xi_{\tilde{N}(w)}^t = y_{\tilde{N}(w)}, \Delta_w^t(\xi_{\tilde{N}(w)}^t) = u_w) \\
Q_W(x_W / y, u) &= \prod_{w \in W} Q_w(x_w / y_{\tilde{N}(w)}, u_w)
\end{align*}
Покладаючи \(W = V\) для \( Q_W(x_W / y, u) \), назвемо ядром переходу
\[Q(x/y,\ u) = P(\xi^{t+1} = x\ |\ \xi^t = y, u^t = u)\]
\end{definition}

Визначений вище керований стохастичний процес \((\xi, \delta)\) відносно графа взаємодії \(G = (V, E)\) утворює керований марківський процес.




%\subsubsection{Задача оптимізації}

Сформулюємо задачу знаходження стратегії, що мінімізує витрати підприємств. Нехай \(E_y^\delta\) – математичне сподівання, що відповідає процесу \((\xi, \delta)\) за початкового стану \(\xi^0 = y\).  Тоді \(C_T^\delta\) — середні очікувані витрати за час \(T\):
\begin{align*}
C_T^{\delta} = &E\left[\frac{1}{T+1}\sum_{t=0}^T r(\xi^t, \Delta^t(\xi^0, \ldots, \xi^t))\right] \\
 = &E_y^\delta\ \frac{1}{T+1}\ \sum_{t=0}^T r(\xi^t, \Delta^t(\xi^0, \ldots, \xi^t)) \\
 = &E_y^\delta\ \frac{1}{T+1}\ \sum_{t=0}^T r(\xi^t, \Delta^t(\xi^{t-1}))
\end{align*}

Задача полягає в знаходженні оптимальної стратегії \(\delta^*\), яка мінімізує \(C_T^{\delta}(y)\) при \(T~\rightarrow~\infty\) для всіх \(y \in X\):
\begin{gather*}
R_y^\delta = \lim\limits_{T \rightarrow \infty} \sup C_T^{\delta}(y) \\
\delta^* = R_y^{\delta^*},\ y \in X
\end{gather*}

Згідно з результатом у \cite{Chornei:2005}, для керованого марківського поля зі скінченними просторами станів та скінченними й інваріантними у часі просторами можливих дій, існує оптимальна стратегія, яка є недетерміністичною, стаціонарною, та марківською. Таким чином, при пошуці оптимальної \(\delta^*\) можна обмежитися лише класом чистих стаціонарних марківських стратегій.

Оскільки кероване марківське поле за умови стаціонарності стратегії утворює однорідний марківський ланцюг, стратегію \(\delta^*\) можна знайти, використовуючи методи лінійного програмування \cite{Knopov:1998}:

\[\min \sum_{x \in X} \sum_{u \in U} r(x, u)\ z_{xu}\]
З обмеженнями:

\begin{gather*}
\sum_{u \in U} z_{xu} = \sum_{y \in X} \sum_{u \in U} Q(x/y, u) z_{yu},\: x \in X \\
\sum_{y \in X} \sum_{u \in U} z_{yu} = 1 \\
z_{xu} \geq 0, x \in X, u \in U
\end{gather*}

де \(z_{xu} = \pi_x D_{xu}\).

\clearpage


% Literature
\nocite{David:1998}
\nocite{Knopov:2011}
\nocite{Knopov:1998}
\nocite{Chornei:2005}
\nocite{Koller:2009}

\bibliography{literature.bib}
\bibliographystyle{plain}

\end{document}
