\documentclass[oneside,draft,14pt]{extarticle}

% Config
% Ukrainian language

\usepackage{fontspec}
\usepackage{polyglossia}
\setmainlanguage{ukrainian}
\setotherlanguage{english}
\newfontfamily\cyrillicfont[Script=Cyrillic]{CMU Serif}
\newfontfamily\cyrillicfontsf[Script=Cyrillic]{CMU Sans Serif}
\newfontfamily\cyrillicfonttt[Script=Cyrillic]{CMU Typewriter Text}

% Page setup
\usepackage{vmargin}
\setpapersize{A4}
\setmarginsrb{2cm}{1.5cm}{1cm}{1.5cm}{0pt}{0mm}{0pt}{13mm}
\renewcommand{\baselinestretch}{1.4}
\sloppy

% Useful packages
\usepackage{amssymb, amsmath, amsfonts, enumerate, float, indentfirst}

% Bibliography
\usepackage{cite}
%\usepackage[nottoc, numbib]{tocbibind}
\makeatletter
\renewcommand{\@biblabel}[1]{#1.}
\makeatother

% Arabic enumerations
\renewcommand{\theenumi}{\arabic{enumi}}
\renewcommand{\labelenumi}{\arabic{enumi}}
\renewcommand{\theenumii}{.\arabic{enumii}}
\renewcommand{\labelenumii}{\arabic{enumi}.\arabic{enumii}.}
\renewcommand{\theenumiii}{.\arabic{enumiii}}
\renewcommand{\labelenumiii}{\arabic{enumi}.\arabic{enumii}.\arabic{enumiii}.}

% Hide sections numbers
% http://tex.stackexchange.com/questions/80113/hide-section-numbers-but-keep-numbering
\renewcommand{\thesection}{}
\renewcommand{\thesubsection}{\arabic{section}.\arabic{subsection}}
\makeatletter
\def\@seccntformat#1{\csname #1ignore\expandafter\endcsname\csname the#1\endcsname\quad}
\let\sectionignore\@gobbletwo
\let\latex@numberline\numberline
\def\numberline#1{\if\relax#1\relax\else\latex@numberline{#1}\fi}
\makeatother

% Theorems
\usepackage{amsthm}
\theoremstyle{definition}
\newtheorem*{definition}{Означення}
\newtheorem*{example}{Приклад}
\newtheorem*{algorithm}{Алгоритм}

\begin{document}

% Title page
\begin{titlepage}
\newpage

\begin{center}
Міністерство освіти і науки України \\
\textsc{Національний Університет «Києво-Могилянська Академія»} \\
Кафедра математики факультету інформатики
\end{center}

\vspace{8em}

\vspace{2em}

\begin{center}
\textsc{\textbf{Застосування Марківських випадкових полів для моделювання економічних процесів з конкурентними технологіями}} \\
Курсова робота за спеціальністю «Прикладна математика»
\end{center}

\vspace{6em}

\begin{flushright}
Керівник курсової роботи \\
к. фіз.-мат. наук, доцент \\
Чорней Р.К. \\
\vspace{2em}
Виконав студент \\
Кулинич Б.В. \\
\vspace{2em}

\end{flushright}

\vspace{\fill}

\begin{center}
Київ – 2014
\end{center}

\end{titlepage}
\setcounter{page}{2}

% Toc
\tableofcontents
\clearpage

% Contents
\section{Процеси з конкурентними категоріями}

\subsection{Постановка задачі}

Описана у цій роботі задача виникає в економіці\cite{David:1998} при моделюванні поведінки систем підприємств, що взаємодіють між собою, причому основним фактором взаємодії є обрані кожним із підприємств технології виробництва. Модель має відповідати на питання, яку технологію слід обирати кожному з підприємств у кожний момент часу, щоб мінімізувати свої витрати і, відповідно, максимізувати дохід. 

У загальному випадку, система описується\cite{Chornei:2005} множиною економічних акторів \(V\), неорієнтованим графом їхніх взаємодій \(G = (V, E)\), та множиною \(X^t = \times_{v \in V}X_v^t,\ X_v^t = \{x^{t,1}_v, x^{t,2}_v, \ldots, x^{t,n_v}_v\}\) можливих станів (технологій виробництва) актора \(v \in V\) у момент часу \(t\).

Надалі, для побудови ефективної моделі, щодо системи буде введено ряд обмежень:
\begin{description} 
	\item[Дискретний час.] Еволюція системи відбуватиметься у дискретному часі, тобто \(t = 0, 1, ...\)
	\item[Синхронність.] Усі актори системи переходитимуть у свій наступний стан одночасно.
	\item[Інваріантність простору станів.] Для всіх \(v \in V\) в кожний момент часу \(t\) \(X_v^t = X_v\)
	\item[Стохастичність.] Для всіх \(v \in V\)
	\[ \xi_v = \{\xi^t_v\ |\ t = 0, 1, \ldots \}\] \[ \xi_v^t : (\Omega, \mathcal{F}, P) \rightarrow X_v,\] 
	де \( \xi_v \) – стохастичний процес, що визначає стан \(v\) у кожен момент часу \(t\)
	\item[Повнота стану.] (умова Маркова) Нехай \(\tilde N(i) = \{j\ |\ (i, j) \in E \} \cup \{i\} \) – підгрупа акторів, що взаємодіють з \(i\), та сам \(i\). Тоді для всіх \(v \in V\):
	\[P(\xi_v^{t+1}\ = x_v\ |\ \xi_v^0 = x^0, \xi_v^1 = x^1, \ldots, \xi_v^t = x^t) = P(\xi_v^{t+1} = x_v\ |\ \xi_{\tilde N(v)}^t = x_{\tilde N(v)}^t )\]
\end{description}

Беручи до уваги цю припущення, можемо змоделювати систему як Марківське випадкове поле з \(\xi_v\) відносно графу взаємодії \(G = (V, E)\).

\subsection{Марківські випадкові поля}

\clearpage

% Literature
\nocite{David:1998}
\nocite{Knopov:2011}
\nocite{Knopov:1998}
\nocite{Chornei:2005}
\nocite{Koller:2009}

\bibliography{literature.bib}
\bibliographystyle{abbrv}

\end{document}
